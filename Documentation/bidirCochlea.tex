
\documentclass{report}
%\usepackage{extsizes}
%\usepackage[latin1]{inputenc}
\usepackage{amsmath,amsfonts,amssymb,cite}
%\usepackage{graphicx,subcaption}
%\usepackage[labelfont=bf]{caption}
\usepackage[portrait,top=0.8in,bottom=1in,left=1in,right=1in]{geometry}





% I found this sqrt code and I like the way the radical appears better. If it gives you any problems just comment it out. -Karl

% New definition of square root:
% it renames \sqrt as \oldsqrt
\let\oldsqrt\sqrt
% it defines the new \sqrt in terms of the old one
\def\sqrt{\mathpalette\DHLhksqrt}
\def\DHLhksqrt#1#2{%
	\setbox0=\hbox{$#1\oldsqrt{#2\,}$}\dimen0=\ht0
	\advance\dimen0-0.2\ht0
	\setbox2=\hbox{\vrule height\ht0 depth -\dimen0}%
	{\box0\lower0.4pt\box2}}








\begin{document}

\begin{align*}
	r_{bm}^*&=\frac{r_{oc}^*}{c_{21}}\sqrt{\Big(\alpha_{oc}+\beta r_{oc}^{*2}\Big)^2+\left(\frac{\Omega}{f}+\delta r_{oc}^{*2}\right)^2}\\\\
	\sin(\psi_{oc}^*)&=\frac{r_{oc}^*\left(\frac{\Omega}{f}+\delta r_{oc}^{*2}\right)}{c_{21}r_{bm}^*}\\\\
	\cos(\psi_{oc}^*)&=\sqrt{1-\Big(\sin(\psi_{oc}^*)\Big)^2}\\\\
	F&=\sqrt{\Big(\alpha_{bm}r_{bm}^*+c_{12}r_{oc}^*\cos(\psi_{oc}^*)\Big)^2+\left(\frac{\Omega}{f}r_{bm}^*+c_{12}r_{oc}^*\sin(\psi_{oc}^*)\right)^2}
\end{align*}
	
	
	
	
	
	
	
	
\end{document}